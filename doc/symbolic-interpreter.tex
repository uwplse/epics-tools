\documentclass[10pt]{article}

\usepackage{url}
\usepackage{hyperref}
\usepackage{relsize}
\usepackage{color}
\usepackage{microtype}
\usepackage{times}
% Set up beautiful source code listings

\usepackage{listings}
\usepackage{color}
\usepackage[scaled]{beramono}
\usepackage[T1]{fontenc}

\definecolor{mygreen}{rgb}{0,0.6,0}
\definecolor{mygray}{rgb}{0.5,0.5,0.5}
\definecolor{mymauve}{rgb}{0.58,0,0.82}

\newcommand{\codehighlight}[1]{\colorbox{yellow}{#1}}
\lstset{
    basicstyle=\small\ttfamily,
    keywordstyle=\color{blue},
    stringstyle=\color{mygreen},
    commentstyle=\color{mygray},
    numberstyle=\color{mymauve},
    showspaces=false,
    showlines=true,
    showstringspaces=false,
    frame=none,
    escapeinside={(*@}{@*)},
    literate=
        {->}{{$\rightarrow$}}2
        {<-}{{$\leftarrow$}}2
        {\\}{{$\lambda$}}1
}

\newcommand{\code}[1]{\lstinline{#1}}


% \|name| or \id{name} denotes math identifiers.
% This gives us proper kerning on multi-character italicised math identifiers.
\def\|#1|{\mathid{#1}}
\newcommand{\mathid}[1]{\ensuremath{\mathit{#1}}}

% \<name> or \codeid{name} denotes computer code identifiers.
\def\<#1>{\codeid{#1}}
\newcommand{\codeid}[1]{\ifmmode{\mbox{\smaller\ttfamily{#1}}}\else{\smaller\ttfamily{#1}}\fi}

% help out listings a bit
\lstdefinelanguage{Rosette}[]{Lisp}{
   morekeywords={assert,define}}

% a few names
\newcommand{\examplefile}{\<properties.rkt>}

\title{EPICS Symbolic Interpreter Manual}
\author{Calvin Loncaric \\ University of Washington}

\begin{document}
\maketitle

\section{Introduction}

This document describes our \emph{symbolic interpreter} for EPICS.%
\footnote{\url{http://www.aps.anl.gov/epics/}} EPICS is an open-source framework
for writing distributed control software. An EPICS program consists of one or
more \emph{IOCs} that interact with hardware devices and communicate with each
other over a network. Our symbolic interpreter can prove strong properties about
an individual IOC---for instance, that no bad values are ever written to a
particular hardware device.

Our symbolic interpreter is especially useful as a \emph{linter} to catch
common syntax errors in EPICS programs and as an \emph{exhaustive tester} that
can check all possible behaviors of an IOC under arbitrary conditions.

\section{Setup}

Our symbolic interpreter requires the following software to be installed:

\begin{itemize}
  \item \textbf{Racket Language (6.4 or higher)} \\
      Debian package: racket \\
      Racket is a general-purpose functional language. At this time Racket 6.4
      has not made it to the Debian repos, so you must remove any existing
      Racket install with \<sudo apt-get remove racket> and acquire the new
      version manually from \url{https://www.racket-lang.org}.

  \item \textbf{Rosette (2.0 or higher)} \\
      Install instructions: \url{https://github.com/emina/rosette} \\
      Rosette is a Racket library for general symbolic interpretation.
      If you had installed a pre-2.0 version of Rosette, you must first remove
      it. Find the path using \<raco link -l | fgrep -i rosette> and remove it
      with \<raco link -r PATH>.
\end{itemize}

\noindent
The symbolic interpreter itself does not require any special installation.

\section{Usage}

To make use of any symbolic interpreter features you must first create a
symbolic version of an existing IOC:
%
\begin{lstlisting}[language=Bash]
    ./bin/setup-symbolic path/to/opIOC iocBoot/iocisoconsole/st.cmd out-dir
\end{lstlisting}
%
This creates a symbolic version of the IOC at \<path/to/opIOC> using the
start command file at \<path/to/opIOC/iocBoot/iocisoconsole/st.cmd>. The
symbolic IOC will be placed in \<out-dir>, possibly overwriting any files
therein.

Notes:
\begin{itemize}
\item
  Re-run \<setup-symbolic> whenever code in the IOC changes.
\item
  Re-run \<setup-symbolic> whenever you upgrade to a new version of this
  symbolic interpreter.
\item
  The \<setup-symbolic> script can take several minutes to run for very large
  IOCs.
\end{itemize}

During setup some basic code linting is performed on the database. This can help
catch typos and other common mistakes before even checking any advanced
properties of the IOC. \autoref{tbl:setup-warnings} lists some common linting
warnings and their meanings.

\begin{table}
  \newcommand{\hole}{\textcolor{blue}{\textbf{??}}}
  \begin{center}
  \begin{tabular}{p{0.33\linewidth}|p{0.66\linewidth}}
    \textbf{Warning} & \textbf{Meaning} \\
    \hline
    Ignored command: \hole{}\ &
      An unsupported command was seen in \<st.cmd>. Usually this is not a
      problem. \\
    truncated \hole{}\ to \hole{}\ in record \hole{}\ &
      The symbolic interpreter does not support floating-point values
      (see \autoref{sec:approximations}). Thus, some values appearing in record
      fields must be truncated to integers. \\
    not sure how to handle EPICS type \hole{}\ &
      The symbolic interpreter tried to export the value of a NOACCESS field to
      the symbolic IOC. This likely indicates a bug in the setup program. \\
    in record \hole{}: cannot coerce \hole{}\ to \hole{}\ &
      Conversions to and from strings are not fully supported. This warning
      indicates that such a conversion is taking place in the given record, and
      may not be modeled correctly. \\
    in record \hole{}: PRINTF ignored &
      The EPICS \<PRINTF> function is not handled by the symbolic interpreter.
      This warning indicates that PRINTF was used in the given record. \\
    unsupported record type in ToRosette.hs: \hole{}\ &
      The given record type is only partially handled by the symbolic interpreter.
      Its \<PACT> and \<FLNK> fields are properly modeled, but other behaviors are
      not. \\
    record \hole{}\ specifies `Use OCAL' but does not specify OCAL &
      The given record is a \<calcout> record that states it should use the
      \<OCAL> expression as output, but does not provide an \<OCAL> expression.
      While EPICS does define behavior in this case, it may indicate a typo in
      the record definition. \\
    omitting duplicate record \hole{}\ &
      After template expansion, a duplicate record with the given name was
      discovered. This almost certainly indicates a bug in the IOC. \\
  \end{tabular}
  \end{center}
  \caption{Common warnings from \<setup-symbolic>.}
  \label{tbl:setup-warnings}
\end{table}

\subsection{Checking properties}

Input properties must be written in Racket. The symbolic interpreter is invoked
by running
%
\begin{lstlisting}[language=Bash]
    ./bin/check-symbolic path/to/symbolic/ioc properties-file.rkt property-name
\end{lstlisting}
%
The file \examplefile{} is an example properties file. A properties file has
the form:
%
\begin{lstlisting}[language=Rosette]
    (define (property1) ...)
    (define (property2) ...)
\end{lstlisting}
%
In general, most properties will look like:
%
\begin{lstlisting}[language=Rosette]
    (define (property-name)
      (define a (get-field "RecordName1" "FIELD1"))
      (define b (get-field "RecordName2" "FIELD2"))
      (process "RecordName3")
      (define x (get-field "RecordName4" "FIELD4"))
      (define y (get-field "RecordName5" "FIELD5"))
      (assert (=>
        (precondition involving a & b)
        (postcondition involving x & y)))
\end{lstlisting}
%
This form is used to verify that if some precondition holds in the database
and ``RecordName3'' gets processed, then some postcondition will always hold
afterwards.

The tool will output either ``everything is ok!'' or ``counterexample.'' In the
latter case, a counterexample will be generated that describes an initial
database state where the given property does not hold.

As a concrete example of an interesting property, here is \<watchDogProp> from
\examplefile{}:
%
\begin{lstlisting}[language=Rosette]
    (define (watchDogProp)
      (define oldval (get-field "CYC:WatchdogAOK:Control" "VAL"))
      (process "CYC:WatchdogAOK:Calc")
      (define newval (get-field "CYC:WatchdogAOK:Control" "VAL"))
      (assert (not (eq? (zero? oldval) (zero? newval)))))
\end{lstlisting}
%
\begin{sloppypar}
This property ensures that every time \<CYC:WatchdogAOK:Calc> gets processed,
\<CYC:WatchdogAOK:Control.VAL> alternates between zero and some non-zero value.
It does this by capturing \<CYC:WatchdogAOK:Control.VAL> before and after
processing \<CYC:WatchdogAOK:Calc> and asserting that the old value is zero if
and only if the new value is NOT zero.
\end{sloppypar}

\subsection{Interactive shell}

Writing and checking and re-checking property files can be tedious. To assist
with this task, the symbolic IOC comes equipped with some helper functions for
interactive development. In the Racket command-line shell or in the Dr. Racket
IDE you can get access to these functions with:
%
\begin{lstlisting}[language=Rosette]
    (require (file "path/to/symbolic/ioc/ishell.rkt"))
\end{lstlisting}
%
In the interactive shell, you can interact with the IOC with the following
procedures:
\begin{itemize}
  \item \lstinline[mathescape]{(examine $\|record|$)} - set the record to examine
  \item \lstinline[mathescape]{(pre $\|expr|$)} - add a new precondition
  \item \lstinline[mathescape]{(post $\|expr|$)} - add a new postcondition
  \item \lstinline{(status)} - show which record is currently being examined as well as the current pre and post conditions
  \item \lstinline{(check)} - verify that whenever the preconditions hold and the current examined record gets processed, then the postconditions hold after processing
  \item \lstinline{(freeze)} - output text that can be used as a property file to check later
  \item \lstinline{(reset)} - reset the state of the shell (i.e.\ clear the current record and all pre and post conditions)
\end{itemize}
%
% Sample interaction:
% %
% \begin{lstlisting}[language=Rosette]
% \end{lstlisting}

% \subsection{Concrete mode}

% [[TODO]]

\section{Supported EPICS Features}

Our symbolic interpreter does not support all of EPICS. Instead, we have focused
on a subset we think will be especially useful. We plan to extend this subset in
the future.

This section assumes intimate knowledge of the standard EPICS record types.%
\footnote{\url{https://wiki-ext.aps.anl.gov/epics/index.php/RRM_3-14}}

\subsection{Approximations}
\label{sec:approximations}

EPICS supports a number of data types including booleans, bitvectors of varying
widths, double-precision floating points, fixed-length strings, fixed-length
arrays over varying types, and several others. Our symbolic interpreter does not
model all of these precisely. Specifically:
\begin{enumerate}
  \item Floats and doubles are modeled as 32- or 64-bit numbers, respectively.
    This means that
    decimal values, division, and other floating-point operations are not
    modeled correctly. However, when the values in the IOC are integers,
    this is an appropriate approximation.
  \item All integral types are modeled exactly.
  \item The EPICS \<MENU> and \<ENUM> types are modeled exactly.
  \item Strings are modeled exactly.
  \item Arrays are modeled exactly (with the above approximations on their
    element types).
\end{enumerate}

\subsection{Record types and fields}

Not all record types and fields are supported. The following records are
supported to some degree. For all records, the behavior of \<PACT> and \<FLNK>
is properly handled. Monitors and device support are not modeled.

\begin{itemize}
  \item \textbf{\<calc>}: fully supported
    argument fetching and calculation (see also \autoref{sec:calc}).
  \item \textbf{\<calcout>}: fully supported
  \item \textbf{\<acalcout>}: fully supported
  \item \textbf{\<scalcout>}: fully supported except for \<PRINTF> and \<SCANF>
  \item \textbf{\<seq>}:
    all \<DLY>$n$ fields are assumed to be 0.
  \item \textbf{\<fanout>}:
    \<SELM> is assumed to be ``All''.
  \item \textbf{\<dfanout>}:
    \<SELM> is assumed to be ``All''.
  \item \textbf{(other)}:
    Other records types are treated as ``dumb data'': the mutable fields listed
    below are included, but the records themselves are treated
    as no-ops during processing. It is still possible to make assertions about
    the \<VAL> fields of these records, but any special behavior is ignored.
    In particular, the \<ai>, \<ao>, \<bi>, \<bo>, \<mbbi>, \<mbbo>, and
    \<waveform> records are in this category.
\end{itemize}

\noindent
Only the following fields are mutable:

\begin{center}
\begin{tabular}{l|l}
(all records) & \<PACT>, \<PROC>, \<VAL>, \<SEVR>, \<STAT> \\
\<calc>       & \<A>..\<L> \\
\<calcout>    & \<A>..\<L> \\
\<acalcout>   & \<A>..\<L>, \<AA>..\<LL>, \<AVAL> \\
\<scalcout>   & \<A>..\<L>, \<AA>..\<LL>, \<SVAL> \\
\<seq>        & \<DO1>..\<DO9>, \<DOA> \\
\end{tabular}
\end{center}

\noindent
Reads of other fields will always return the \emph{default value} for the field:
the value provided in the IOC source code (if present) or the default
value for the field otherwise. Writes to unsupported fields are treated as
no-ops and a warning is written to the log.

\subsection{Alarm handling}

Alarms are handled for the analog records \<calc>, \<calcout>, \<acalcout>,
\<scalcout>, \<ao>, and \<ai>, but alarm hysteresis is not handled. Alarms for
binary records such as \<bi> and \<mbbo> are not handled. Alarm propagation
along links and the ``Maximize Severity'' link attribute are fully supported.

\subsection{Calculations}
\label{sec:calc}

Not all features of the calculation records \<calc>, \<calcout>, \<acalcout>,
and \<scalcout> are supported. Supported syntax:

\begin{center}
\begin{tabular}{ll|l}
$x$                 & variables        & \\
1.0                 & numeric values   & rounded to nearest integer \\
``x''               & strings          & \\
$f(x)$              & function calls   & partial support; see below \\
$-x, \sim{x}, !x$   & unary operators  & \\
$x + y, x - y, ...$ & binary operators & \\
$x ? y : z$         & ternary operator & \\
$x := y$            & assignment & \\
$x; y$              & sequence & \\
$x[y,z]$            & subarray & \\
$x\{y,z\}$          & subarray in-place & old semantics; see below \\
\end{tabular}
\end{center}

\noindent
Note that in the current implementation of the symbolic interpreter, subarray
in-place has the \emph{old} semantics where the bounds are inclusive-exclusive.
This differs from the semantics of the plain subarray operator that has
inclusive-inclusive bounds. This behavior changed in version 3-0 of the calc
support library%
\footnote{\url{http://www.aps.anl.gov/bcda/synApps/calc/calc.html}} so that both
operators now have inclusive-inclusive bounds.

Supported functions:

\begin{center}
\begin{tabular}{l|l}
  \<MIN>, \<MAX> & \\
  \<ABS> & \\
  \<ARR> & \\
  \<FLOOR> & \\
  \<PRINTF>, \<SSCANF> & ignored \\
  (other) & assumed to return 0 \\
\end{tabular}
\end{center}

\section{Additional Resources}

\begin{itemize}
\item \textbf{The Rosette Guide}: \url{https://emina.github.io/rosette/rosette-guide/index.html}
\item \textbf{EPICS Record Reference Manual (for version 3.14)}: \url{https://wiki-ext.aps.anl.gov/epics/index.php/RRM_3-14}
\end{itemize}

\end{document}
